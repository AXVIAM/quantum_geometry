**Quantum Information Gravity: Model Equations & Derivations**

**Objective:** Provide a full breakdown of the equations used in QIR’s gravitational lensing, black hole entropy, and quantum uncertainty models, including derivations, theoretical justifications, and correction terms.

---

## **1. Gravitational Lensing Equation**

### **Standard Lensing Equation (Einstein’s Prediction)**
\[
\theta_E = \sqrt{\frac{4GM}{c^2 D}}
\]
where:
- \( G \) is the gravitational constant
- \( M \) is the mass of the lens
- \( c \) is the speed of light
- \( D \) is the distance between the observer and the lens

### **QIR-Modified Lensing Equation**
\[
\theta_{QIR} = \sqrt{\frac{4GM}{c^2 D}} + \alpha I(x)
\]
where:
- \( \alpha \) is the QIR correction factor
- \( I(x) \) represents the information density at a given point in spacetime

**Implication:** The additional term suggests that lensing effects increase slightly due to information-theoretic modifications to curvature.

---

## **2. Black Hole Entropy Equation**

### **Standard Bekenstein-Hawking Entropy**
\[
S_{BH} = \frac{k_B A}{4 L_p^2}
\]
where:
- \( S_{BH} \) is the classical black hole entropy
- \( k_B \) is Boltzmann’s constant
- \( A \) is the area of the event horizon
- \( L_p \) is the Planck length

### **QIR-Modified Entropy Equation**
\[
S_{QIR} = S_{BH} \left(1 + \beta \log(M/M_0)\right)
\]
where:
- \( \beta \) is the correction factor derived from QIR principles
- \( M_0 \) is a normalization mass scale

**Implication:** The QIR correction introduces an additional entropy term that grows logarithmically with mass, explaining why entropy deviations increase for larger black holes.

---

## **3. Quantum Uncertainty Equation**

### **Standard Heisenberg Uncertainty Principle**
\[
\Delta x \Delta p \geq \frac{\hbar}{2}
\]
where:
- \( \Delta x \) is the position uncertainty
- \( \Delta p \) is the momentum uncertainty
- \( \hbar \) is the reduced Planck constant

### **QIR-Modified Uncertainty Relation**
\[
\Delta x \geq \frac{\hbar}{m c} \frac{1}{I(x)}
\]
where:
- \( I(x) \) represents the local information density
- \( m \) is the particle mass

**Implication:** At small scales, information density alters uncertainty constraints, potentially explaining deviations in quantum fluctuation measurements.

---

## **4. Refinements & Next Steps**
- Investigate scale-dependent modifications to \( \alpha \) and \( \beta \) to refine deviations in observational data.
- Explore additional observational datasets to confirm the patterns seen in gravitational lensing, black hole entropy, and quantum uncertainty.

**This document will be continuously updated as refinements are made to QIR’s equations.**

