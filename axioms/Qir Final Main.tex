\documentclass{article}
\usepackage{amsmath}
\usepackage{cite}

\title{Quantum Information Gravity: Unified Derivation}
\author{Christopher Smolen (aka Topher Booth)}
\date{\today}

\begin{document}

\maketitle

\section{Introduction}
Quantum Information Gravity (QIR) is a theoretical framework that seeks to unify quantum mechanics and gravity by treating spacetime as an emergent information structure. Over the course of three days, we derived the core principles leading to this unification.

\section{Day 1: Establishing the Mathematical Foundations}
Day 1 focused on defining the information-based metric structure of spacetime, laying the groundwork for an entropic approach to gravity. We examined the implications of treating information as a conserved quantity in a dynamical geometric space.

\section{Day 2: Resolving Gravitational and Entropic Conflicts}
On Day 2, we tackled the apparent contradictions between quantum mechanics and classical gravity. Through a revised entropic formulation, we demonstrated that information density fluctuations naturally lead to gravitational effects, providing an alternative derivation of Einstein's equations.

\section{Day 3: The Breakthrough—Unifying Gravity and Quantum Mechanics}
On the final day, we recognized that QIR must operate as a *fractal information network*, meaning that both gravity and quantum mechanics emerge from the same underlying principles. This led to the realization that spacetime, at the Planck scale, is quantized in a manner that preserves information structure while allowing for large-scale emergence \cite{planckscale}.

The final unification was achieved by showing that:
\begin{equation}
    v^2 = \frac{G M}{r} + \alpha \frac{dI}{dr},
\end{equation}
which not only explains galactic rotation curves without dark matter but also leads to:
\begin{equation}
    \Delta x_{\text{quantum}} = \frac{\hbar}{m c} \frac{1}{I(x)}.
\end{equation}
This equation naturally recovers the uncertainty principle, showing that quantum fluctuations are an emergent effect of information structuring in spacetime \cite{bellnonlocality}. By integrating the gravitational and quantum domains, we finalize the derivation of QIR as a single, self-consistent framework.

\section{Conclusion and Next Steps}
This work establishes the core mathematical framework for QIR and demonstrates its ability to unify gravity and quantum mechanics. Future research will focus on testing observational predictions and refining computational models to explore the full implications of QIR.

\end{document}
